\section{Diskussion}
\label{sec:Diskussion}
Die reale Gütezieffer weicht stark von der idealen ab, dies lässt sich durch mehrere Annahmen begründen.
Als erstes ist das System nicht vollständig abgeschlossen und es findet konstant ein Austausch mit der UMgebung statt.
Desweiteren kann angenommen werden, dass der Kompressor nicht vollständig adiabatisch arbeitet.
Eine technische Umsetzung einer komplett adiabatischen Zustandsänderung ist nach dem zweiten Hauptsatz der Thermodynamik fast ausgeschlossen.
Zusaätzlich erzeugt der Kompressor nicht nur ein Druckgefälle sonder auch deutlich messbare mechanische Schwingungen.
Aus diesen Annahmen ist erkennbar, dass der Kompressor erheblichen Einfluss auf die Qualität der Güteziffer hat.
