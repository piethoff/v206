\section{Zielsetzung}
Untersuchung einer Wärmepumpe und ermittlung der Qualität dieser.
\section{Theorie}
\label{sec:Theorie}

Der zweite Hauptsatz der Thermodynamik besagt, dass Wärmeenergie vom wärmeren in das kältere Reservoir fließt.
Damit sich die Flussrichtung der WäWärmeenergie ändert muss beispielsweise mechanische Arbeit aufgewendet werden.
Eine Apparatur, welche dieses vollführt, ist die Wärmepume.
Die Güteziffer

\begin{equation*}
	\nu=\frac{Q_1}{A}
\end{equation*}

beschreibt das Verhältnis zwischen transportierter Wärmemenge und verrichteter Arbeit.
Mit dem zweiten Hauptsatz der Thermodynamik lässt sich nun folgende Aussage über über Wärmemengen der beiden Reservoirs und Temperaturen dieser treffen:
\begin{equation*}
	\frac{Q_1}{T_1}-\frac{Q_2}{T_2}=0.
\end{equation*}
Dies gilt jedoch nur unter idealisierten den Bedingungen, dass der Prozess reversibel ist.
Hieraus lässt sich für die Güteziffer die Beziehung
\begin{equation}
	\nu_{\text{id}}=\frac{T_1}{T_1-T_2}
	\label{eq:guetezifferideal}
\end{equation}
folgern.
Die reale Wärmepumpe kann diese forderung jedoch nicht erfüllen.
Für den realen, irreversiblen Fall gilt
\begin{equation*}
	\frac{Q_1}{T_1}-\frac{Q_2}{T_2}>0 .
\end{equation*}
Daraus lässt sich folgern, dass die Wärmepumpe um so günstiger arbeitet je kleiner die Temperaturdifferenz zwischen den Wärmereservoirs ist.
