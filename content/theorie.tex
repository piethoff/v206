\section{Zielsetzung}
Untersuchung einer Wärmepumpe und ermittlung der Qualität dieser.
\section{Theorie}
\label{sec:Theorie}

Der zweite Hauptsatz der Thermodynamik besagt, dass Wärmeenergie vom wärmeren in das kältere Reservoir fließt.
Damit sich die Flussrichtung der WäWärmeenergie ändert muss beispielsweise mechanische Arbeit aufgewendet werden.
Eine Apparatur, welche dieses vollführt, ist die Wärmepume.
Die Güteziffer

\begin{equation}
	\label{eq:gl1}
	\nu=\frac{Q_1}{A}
\end{equation}

beschreibt das Verhältnis zwischen transportierter Wärmemenge und verrichteter Arbeit.
Mit dem zweiten Hauptsatz der Thermodynamik lässt sich nun folgende Aussage über über Wärmemengen der beiden Reservoirs und Temperaturen dieser treffen:
\begin{equation}
		\label{eq:gl2}
	\frac{Q_1}{T_1}-\frac{Q_2}{T_2}=0.
\end{equation}
Dies gilt jedoch nur unter idealisierten den Bedingungen, dass der Prozess reversibel ist.
Hieraus lässt sich für die Güteziffer die Beziehung
\begin{equation}
	\nu_{\text{id}}=\frac{T_1}{T_1-T_2}
	\label{eq:guetezifferideal}
\end{equation}
folgern.
Die reale Wärmepumpe kann diese forderung jedoch nicht erfüllen.
Für den realen, irreversiblen Fall gilt
\begin{equation}
		\label{eq:gl3}
	\frac{Q_1}{T_1}-\frac{Q_2}{T_2}>0 .
\end{equation}
Daraus lässt sich folgern, dass die Wärmepumpe um so günstiger arbeitet je kleiner die Temperaturdifferenz zwischen den Wärmereservoirs ist.
\subsection{Aufbau der Wärmepumpe}
\label{sec:AdW}
In Abbildung wird der Aufbau einer Wärmepumpe dargestellt.
Innerhalb eines geschlossenen Systems verdampft und kondensiert ein reales Gas in den verschiedenen Wärmereservoirs.
Dies wird durch Druckunterschiede in den Wärmeresrvoirs realisiert.
Das Gas wird komprimiert wodurch es kondensiert und Wärme an das Reservoir 1 abgibt.
Das Transportmedium durchläuft das Drosselventil D und geht in dem Reservoir 2 in Gasform über.
Hierbei entzieht es Reservoir 2 Wärme.
In dem flüssigen Zustand wird das Transportmedium von Gasresten mittels eines "Reinigers" befreit, um das Drosselventil nicht zu beschädigen.
\subsection{Bestimmung der Kenngrößen}
\label{sec:BdK}
