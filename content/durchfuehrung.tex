\section{Durchführung}
\label{sec:Durchführung}
Der Versuch wird nach Abbildung \ref{fig:aufbau} aufgebaut.
Die Wärmereservoirs werden jeweils mit $\SI{3}{\liter}$ Wasser befüllt, welches zu Beginn des Versuchs Raumtemperatur besitzt.
Das Wasser in den Reservoirs wird von Motoren dauerhaft in Bewegung gehalten, um eine homogene Wärmeverteilung zu erreichen.
Die Temperaturen werden mit Thermometern bestimmt.
Die Drücke können an zwei Manometern abgelesen werden und die Leistung des Kompressors an einem Wattmeter.
Die so erhaltenen Werte werden jede Minute notiert.
Der Versuch wird beendet bevor $T_1$ eine Temperatur von $\SI{50}{\celsius}$ erreicht.
