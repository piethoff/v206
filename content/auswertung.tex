\section{Auswertung}
\label{sec:Auswertung}
\subsection{Temperaturverlauf}
Die gemessenen Werte sind in einem geeigneten Diagramm \ref{fig:tempfit} dargestellt.
Die Ausgleichsrechnung wurde in diesem Diagramm bereits aufgetragen.
\begin{table}[H]
    \centering
    \caption{Messwerte des Versuchs (ohne 1 Bar Umgebungsdruck).}
    \label{tab:t1}
    \sisetup{table-format=2.2}
    \begin{tabular}{S[table-format=2.0] S S S S S[table-format=3.0]}
        \toprule
        {Zeit$/\si{\minute}$} & {$T_1/\si{\celsius}$} & {$T_2/\si{\celsius}$} & {$p_a/\si{\bar}$} & {$p_b/\si{\bar}$} & {$A/\si{\watt}$}\\
        \midrule
        1   & 21.6  & 21.0  & 1.4   & 6     & 160   \\
        2   & 22.3  & 21.0  & 1.6   & 6.2   & 175   \\
        3   & 23.5  & 20.0  & 1.82  & 7.0   & 180   \\
        4   & 25.0  & 18.7  & 2.0   & 7.0   & 195   \\
        5   & 26.8  & 17.0  & 2.0   & 7.2   & 197   \\
        6   & 28.5  & 15.2  & 2.1   & 8.0   & 201   \\
        7   & 30.4  & 13.4  & 2.1   & 8.1   & 202   \\
        8   & 32.4  & 11.5  & 2.2   & 8.5   & 205   \\
        9   & 34.7  & 9.7   & 2.2   & 9     & 210   \\
        10  & 35.9  & 8.0   & 2.2   & 9.5   & 210   \\
        11  & 37.8  & 6.3   & 2.2   & 9.6   & 212   \\
        12  & 39.3  & 4.7   & 2.2   & 10.1  & 210   \\
        13  & 40.8  & 3.3   & 2.2   & 10.5  & 212   \\
        14  & 42.2  & 2.0   & 2.2   & 11.0  & 214   \\
        15  & 44.1  & 0.9   & 2.2   & 11.2  & 215   \\
        16  & 45.7  & 0.2   & 2.2   & 11.5  & 215   \\
        17  & 47.2  & -0.3  & 2.2   & 11.6  & 210   \\
        18  & 48.6  & -0.7  & 2.2   & 12    & 210   \\
        \bottomrule
    \end{tabular}
\end{table}
\subsection{Ausgleichsrechnung}
Als eine geeignete Näherung für die Temperaturverläufe wurde
\begin{equation}
  \label{eq:tempfit}
  T(t) = At^2 + Bt + C
\end{equation}
gewählt.
\begin{figure}[H]
  \centering
  \includegraphics[scale=0.5]{tempfit.pdf}
  \caption{Temperaturen der Wärmereservoirs aufgetragen gegen die Zeit.}
  \label{fig:tempfit}
\end{figure}
Dabei ergeben sich für die Koeffizienten $A$, $B$ und $C$:
\begin{equation*}
    A_{T1} =\SI{-0.1\pm 1.2e-6}{\kelvin\per\second\squared}
\end{equation*}
\begin{equation*}
    B_{T1} =\SI{0.028\pm 0.0014}{\kelvin\per\second}
\end{equation*}
\begin{equation*}
    C_{T1} =\SI{292.05\pm 0.34}{\kelvin}
\end{equation*}
\begin{equation*}
    A_{T2} =\SI{5.9\pm 2.4e-6}{\kelvin\per\second\squared}
\end{equation*}
\begin{equation*}
    B_{T2} =\SI{-0.0312\pm 0.0029}{\kelvin\per\second}
\end{equation*}
\begin{equation*}
    C_{T2} =\SI{298.0\pm 0.7}{\kelvin}
\end{equation*}
Die Differentialquotienten $\symup{d}T_1/\symup{dt}$ und $\symup{d}T_2/\symup{dt}$ können mithilfe der Näherungsfunktion
bestimmt werden:
\begin{equation}
    \frac{\symup{d}T_1}{\symup{dt}} = 2At + B
\end{equation}
Es ergeben sich folgende Werte für 4 verschiedene, gewählte Temperaturen:
\begin{table}[H]
    \centering
    \caption{Differentialquotienten von $T_1$ und $T_2$.}
    \label{tab:t2}
    \sisetup{table-format=3.4(4)e1}
    \begin{tabular}{S[table-format=2.0] S[table-format=3.2] S[table-format=1.4(4)e1] S[table-format=3.2] S[table-format=3.4(4)e1]}
        \toprule
        {Zeit$/\si{\minute}$} & {$T_1/\si{\kelvin}$} & {$\frac{\symup{d}T_1}{\symup{dt}}$} & {$T_2/\si{\kelvin}$} & {$\frac{\symup{d}T_2}{\symup{dt}}$} \\
        \midrule
        3   & 296.65    & 0.028 \pm 0.0042  &  293.15 & -0.031 \pm 0.0087\\
        7   & 303.55    & 0.028 \pm 0.0098 &  286.55 & -0.031 \pm 0.0203 \\
        11  & 310.95    & 0.028 \pm 0.0154 &  279.45 & -0.031 \pm 0.0315\\
        15  & 317.25    & 0.028 \pm 0.021 &  274.05 & -0.031 \pm 0.0435\\
        \bottomrule
    \end{tabular}
\end{table}
%
\subsection{Güteziffer}
Mithilfe der Gleichungen \eqref{eq:gueteziffer} und \eqref{eq:leistung} kann nun die Gütezahl der realen Wärmepumpe berechnet
werden:
\begin{equation}
    \nu_\text{real} = \frac{m_wc_w + m_sc_s}{N}\frac{\symup{d}T}{\symup{dt}}
\end{equation}
Hier ist $A$ die angezeigte Leistung des Kompressors, $m_sc_s$ die Wärmekapazität des Systems angegeben mit
$\SI{660}{\joule\per\kelvin}$ und $m_wc_w$ die Wärmekapazität des Wassers in einem der Reservoire, die mit dem Literaturwert
von $c_w = \SI{4182}{\joule\per\kilogram\per\kelvin}$ \cite{const} bestimmt wird.
Die so erhaltenen Güteziffern befindent sich in Tabelle \ref{tab:guete}.
Dazu sind noch die idealen Güteziffern und Abweichung von diesen Aufgetragen.
\begin{table}[H]
    \centering
    \caption{Ergebnisse für reale und ideale Güteziffern}
    \label{tab:guete}
    \sisetup{table-format=2.4}
    \begin{tabular}{S[table-format=2.0] S[table-format=2.4(4)e1] S S}
        \toprule
        {Zeit$/\si{\minute}$} & {$\rho_{real}$} & {$\rho_{ideal}$} & {$Abweichung$} \\
        \midrule
        3  & 0.205 \pm 0.0042 & 84.76 & 99.76\% \\
        7  & 0.183 \pm 0.0098 & 17.86 & 98.97\% \\
        11 & 0.174 \pm 0.0154 & 9.87 & 98.23\% \\
        15 & 0.172 \pm 0.021 & 7.34 & 97.66\% \\
        \bottomrule
    \end{tabular}
\end{table}


%
\subsection{Massendurchsatz}
Der Massenstrom lässt sich mit Gleichung \eqref{eq:massendurchsatz} berechnen, jedoch wird dazu die Verdampfungswärme
$\symup{L}$ benötigt.
Aus der Anleitung des Versuchs \href{http://129.217.224.2/HOMEPAGE/PHYSIKER/BACHELOR/AP/SKRIPT/V203.pdf}{v203} erhält man
folgende Gleichung:
\begin{equation}
    \ln \left(p\right) = -\frac{\symup{L}}{\symup{R}} \frac{1}{T} + \text{const.}
\end{equation}
In folgendem halblogarithmischen Diagramm, ist eine Ausgleichsgerade einfegügt aus dessen Steigung sich L berechnen lässt.
\begin{figure}[H]
\centering
\includegraphics[width=\textwidth]{build/dampfdruck.pdf}
\caption{Dampfdruckkurve von $\ce{Cl2F2C}$.}
\end{figure}
%
Aus dieser Gleichung und den Messreihen für $p_b$ und $T_1$ kann mit der allgemeinen Gaskonstante $R=\SI{8.31}{\joule\per\mole\per\kelvin}$
die Verdampfungswärme $L=\SI{}{\joule\per\mole}=\SI{}{\joule\kilogram}$, mit der molaren Masse von $\ce{Cl2F2C}$
%
$\SI{120.91}{\gram\per\mole}$ \cite{molar}, bestimmt werden.

\subsection{Mechanische Kompressorleistung}
Abschließend lässt sich die mechanische Kompressorleistung über Gleichung \eqref{eq:arbeit} bestimmen.
Die Dichte von $\ce{Cl2F2C}$ bei $T=\SI{0}{\celsius}$ und $p=\SI{1}{\bar}$ liegt bei $\rho_0=\SI{5.51}{\gram\per\liter}$ und
$\kappa$ ist gegeben mit $\SI{1.14}{}$.
Somit ist $N_\text{mech}=\SI{}{\watt}$.
%
