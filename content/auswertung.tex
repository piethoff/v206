\section{Auswertung}
\label{sec:Auswertung}
\subsection{Temperaturverlauf}
Die gemessenen Werte sind in einer geeigneten Abbildung \ref{fig:tempfit} dargestellt.
Die Ausgleichsrechnung ist in diesem Diagramm bereits aufgetragen.
\begin{table}[H]
    \centering
    \caption{Messwerte des Versuchs (ohne 1 Bar Umgebungsdruck).}
    \label{tab:t1}
    \sisetup{table-format=2.2}
    \begin{tabular}{S[table-format=2.0] S S[table-format=+2.2] S S S[table-format=3.0]}
        \toprule
        {Zeit$/\si{\minute}$} & {$T_1/\si{\celsius}$} & {$T_2/\si{\celsius}$} & {$p_a/\si{\bar}$} & {$p_b/\si{\bar}$} & {$A/\si{\watt}$}\\
        \midrule
        1   & 21.6  & 21.0  & 1.4   & 6.0   & 160   \\
        2   & 22.3  & 21.0  & 1.6   & 6.2   & 175   \\
        3   & 23.5  & 20.0  & 1.8   & 7.0   & 180   \\
        4   & 25.0  & 18.7  & 2.0   & 7.0   & 195   \\
        5   & 26.8  & 17.0  & 2.0   & 7.2   & 197   \\
        6   & 28.5  & 15.2  & 2.1   & 8.0   & 201   \\
        7   & 30.4  & 13.4  & 2.1   & 8.1   & 202   \\
        8   & 32.4  & 11.5  & 2.2   & 8.5   & 205   \\
        9   & 34.7  & 9.7   & 2.2   & 9.0   & 210   \\
        10  & 35.9  & 8.0   & 2.2   & 9.5   & 210   \\
        11  & 37.8  & 6.3   & 2.2   & 9.6   & 212   \\
        12  & 39.3  & 4.7   & 2.2   & 10.1  & 210   \\
        13  & 40.8  & 3.3   & 2.2   & 10.5  & 212   \\
        14  & 42.2  & 2.0   & 2.2   & 11.0  & 214   \\
        15  & 44.1  & 0.9   & 2.2   & 11.2  & 215   \\
        16  & 45.7  & 0.2   & 2.2   & 11.5  & 215   \\
        17  & 47.2  & -0.3  & 2.2   & 11.6  & 210   \\
        18  & 48.6  & -0.7  & 2.2   & 12.0  & 210   \\
        \bottomrule
    \end{tabular}
\end{table}
\subsection{Ausgleichsrechnung}
Als geeignete Näherung für die Temperaturverläufe wird
\begin{equation}
    T \approx F(t) = At^3 + Bt^2 + Ct + D
\end{equation}
gewählt.
\begin{figure}[H]
  \centering
  \includegraphics[scale=0.5]{tempfit.pdf}
  \caption{Temperaturen der Wärmereservoirs aufgetragen gegen die Zeit.}
  \label{fig:tempfit}
\end{figure}
\noindent Dabei ergeben sich für die Koeffizienten $A$, $B$, $C$ und $D$:
\begin{align*}
    A_{T_1} & = \SI[per-mode=reciprocal]{-1.31 \pm 0.28e-8}{\kelvin\per\second\cubed}&    \\
    B_{T_1} & = \SI[per-mode=reciprocal]{2.2 \pm 0.5e-5}{\kelvin\per\second\squared} &          \\
    C_{T_1} & = \SI[per-mode=reciprocal]{0.018 \pm 0.002}{\kelvin\per\second}        &               \\
    D_{T_1} & = \SI[per-mode=reciprocal]{293.18 \pm 0.33}{\kelvin}                   &    \\
\\
    A_{T_2} & = \SI[per-mode=reciprocal]{3.39 \pm 0.23e-8}{\kelvin\per\second\cubed} &   \\
    B_{T_2} & = \SI[per-mode=reciprocal]{-5.2 \pm 0.4e-5}{\kelvin\per\second\squared}&           \\
    C_{T_2} & = \SI[per-mode=reciprocal]{-0.004 \pm 0.002}{\kelvin\per\second}       &                \\
    D_{T_2} & = \SI[per-mode=reciprocal]{295.05 \pm 0.27}{\kelvin}                   &    \\
\end{align*}
Die Differentialquotienten $\symup{d}T_1/\symup{dt}$ und $\symup{d}T_2/\symup{dt}$ können mithilfe der Näherungsfunktion
bestimmt werden:
\begin{equation}
    \frac{\symup{d}T}{\symup{dt}} \approx 3At^2 + 2Bt + C
\end{equation}
Die Unsichertheit ergibt sich mit der Gaußschen Fehlerfortpflanzung zu:
\begin{equation}
    \symup{\Delta}\frac{\symup{d}T}{\symup{dt}} = \sqrt{\left(3t^2\symup{\Delta}A\right)^2+\left(2t\symup{\Delta}B\right)^2+\left(\symup{\Delta}C\right)^2}
\end{equation}
Es ergeben sich folgende Werte für 4 verschiedene, gewählte Temperaturen:
\begin{table}[H]
    \centering
    \caption{Differentialquotienten von $T_1$ und $T_2$.}
    \label{tab:t2}
    \sisetup{table-format=3.4(4)e1}
    \begin{tabular}{S[table-format=2.0] S[table-format=3.2] S[table-format=2.2(4)] S[table-format=3.2] S[table-format=+2.2(3)]}
        \toprule
        {Zeit$/\si{\minute}$} & {$T_1/\si{\kelvin}$} & {$\frac{\symup{d}T_1}{\symup{dt}}/\SI{e-3}{\kelvin\per\second}$} & {$T_2/\si{\kelvin}$} & {$\frac{\symup{d}T_2}{\symup{dt}}/\SI{e-3}{\kelvin\per\second}$} \\
        \midrule
        3   & 296.65    & 24.65 \pm  2.70 & 293.15 & -19.42 \pm 2.47\\
        7   & 303.55    & 29.55 \pm  4.88 & 286.55 & -29.74 \pm 4.10\\
        11  & 310.95    & 29.92 \pm  7.81 & 279.45 & -28.34 \pm 6.40\\
        15  & 317.25    & 25.77 \pm 11.46 & 274.05 & -15.22 \pm 9.33\\
        \bottomrule
    \end{tabular}
\end{table}
%
\subsection{Güteziffer}
Mithilfe der Gleichungen \eqref{eq:gueteziffer} und \eqref{eq:leistung} kann nun die Gütezahl der realen Wärmepumpe berechnet
werden:
\begin{equation}
    \nu_\text{real} = \frac{m_wc_w + m_sc_s}{N}\frac{\symup{d}T_1}{\symup{dt}}
\end{equation}
Die Unsicherheit ist bestimmt durch:
\begin{equation}
    \symup{\Delta}\nu_\text{real} = \frac{m_wc_w + m_sc_s}{N}\symup{\Delta}\frac{\symup{d}T_1}{\symup{dt}}
\end{equation}
Hier ist $N$ die angezeigte Leistung des Kompressors, $m_sc_s$ die Wärmekapazität des Systems angegeben mit
$\SI{660}{\joule\per\kelvin}$ und $m_wc_w$ die Wärmekapazität des Wassers in einem der Reservoire, die mit dem Literaturwert
von \mbox{$c_w = \SI{4182}{\joule\per\kilogram\per\kelvin}$\cite{const}} bestimmt wird.
Die so erhaltenen Güteziffern befinden sich in Tabelle \ref{tab:guete}.
Dazu sind die idealen Güteziffern und Abweichung von diesen aufgetragen.
\begin{table}[H]
    \centering
    \caption{Ergebnisse für reale und ideale Güteziffern.}
    \label{tab:guete}
    \sisetup{table-format=2.2}
    \begin{tabular}{S[table-format=2.0] S[table-format=1.2(3)] S S}
        \toprule
        {Zeit$/\si{\minute}$} & {$\nu_\text{real}$} & {$\nu_\text{ideal}$} & {Abweichung$/\si{\percent}$} \\
        \midrule
        3  & 1.81 \pm 0.20 & 84.76 & 97.86 \\
        7  & 1.93 \pm 0.32 & 17.86 & 89.19 \\
        11 & 1.86 \pm 0.49 & 9.87  & 81.16 \\
        15 & 1.58 \pm 0.70 & 7.34  & 78.47 \\
        \bottomrule
    \end{tabular}
\end{table}
%
\subsection{Massendurchsatz}
Der Massendurchsatz lässt sich mit Gleichung \eqref{eq:massendurchsatz} berechnen, jedoch wird dazu die Verdampfungswärme
$L$ benötigt.
In der Anleitung des Versuchs \href{http://129.217.224.2/HOMEPAGE/PHYSIKER/BACHELOR/AP/SKRIPT/V203.pdf}{v203} findet sich
folgende Gleichung:
\begin{equation}
    \ln \left(p\right) = -\frac{L}{R} \frac{1}{T} + \text{const.}
\end{equation}
In folgendem halblogarithmischen Diagramm, in welchem $\ln{p}$ gegen $1/T$ 
aufgetragen ist, ist eine Ausgleichsgerade eingefügt aus dessen Steigung $m$ sich $L$ berechnen 
lässt.
\begin{equation}
    L = -mR
\end{equation}
Die Unsicherheit ist somit:
\begin{equation}
    \symup{\Delta}L = -R\symup{\Delta}m
\end{equation}
\begin{figure}[H]
    \centering
    \includegraphics[width=\textwidth]{build/dampfdruck.pdf}
    \caption{Dampfdruckkurve von $\ce{Cl2F2C}$.}
\end{figure}
%
\noindent Aus dieser Gleichung und den Messreihen für $p_b$ und $T_1$ kann mit der allgemeinen 
Gaskonstante \mbox{$R=\SI{8.31}{\joule\per\mole\per\kelvin}$}, mit der molaren Masse von $\ce{Cl2F2C}$ \mbox{$\SI{120.91}{\gram\per\mole}$\cite{molar}} 
und \mbox{$m = \SI{2081.94\pm65.97}{\kelvin}$} die Verdampfungswärme
\begin{equation*}
L=\SI{17310\pm549}{\joule\per\mole}=\SI{143166\pm4536}{\joule\per\kilogram},
\end{equation*}
bestimmt werden.
%
Der Massendurchsatz kann nun mithilfe von Gleichung \eqref{eq:gueteziffer} und \eqref{eq:massendurchsatz} 
zu
\begin{equation}
    \frac{\symup{d}m}{\symup{dt}} = \frac{1}{L}(m_2c_w + m_sc_s)\frac{\symup{d}T_2}{\symup{dt}}
\end{equation}
bestimmt werden mit der Unsicherheit
\begin{equation}
    \symup{\Delta}\frac{\symup{d}m}{\symup{dt}} = 
    \sqrt{
        \left(\frac{1}{L^2}(m_2c_w + m_sc_s)\frac{\symup{d}T_2}{\symup{dt}}\symup{\Delta}L\right)^2 + 
        \left(\frac{1}{L}(m_2c_w + m_sc_s)\symup{\Delta}\frac{\symup{d}T_2}{\symup{dt}}\right)^2
    }
\end{equation}
und sind im Folgenden aufgetragen:
\begin{table}[H]
    \centering
    \caption{Massendurchsatz.}
    \label{tab:masse}
    \sisetup{table-format=+1.2(4)}
    \begin{tabular}{S[table-format=2.0] S[table-format=+2.2(3)] S}
        \toprule
        {Zeit$/\si{\minute}$} & {$\frac{\symup{d}T_2}{\symup{dt}}/\SI{e-3}{\kelvin\per\second}$} & {$\frac{\symup{d}m}{\symup{dt}}/\SI{e-3}{\kg\per\second}$} \\
        \midrule
        3   & -19.42 \pm  2.47   & -1.79 \pm 0.235\\
        7   & -29.74 \pm  4.10   & -2.74 \pm 0.388\\
        11  & -28.34 \pm  6.40   & -2.61 \pm 0.590\\
        15  & -15.22 \pm  9.33   & -1.40 \pm 0.862\\
        \bottomrule
    \end{tabular}
\end{table}
%
\subsection{Mechanische Kompressorleistung}
Abschließend lässt sich die mechanische Kompressorleistung über Gleichung \eqref{eq:arbeit} bestimmen.
Die Dichte von $\ce{Cl2F2C}$ bei \mbox{$T_0=\SI{0}{\celsius}$} und \mbox{$p_0=\SI{1}{\bar}$} liegt bei 
\mbox{$\rho_0=\SI{5.55}{\kg\meter\cubed}$\cite{molar}} und
$\kappa$ ist gegeben mit $\num{1.14}$.
Aus der allgemeinen Gasgleichung\cite{gasgl},
\begin{equation}
    \frac{p}{\rho T} = R,
\end{equation}
folgt:
\begin{equation}
    \frac{1}{\rho} = \frac{p_0}{p_a} \frac{T_2}{T_0} \frac{1}{\rho_0}
\end{equation}
Es ergibt sich für die Mechanische Leistung:
Die Unsicherheit ist gegeben nach:
\begin{equation}
    N_{\text{mech}} = \frac{1}{\kappa - 1} \left( p_{\text{b}}
        \sqrt[{\leftroot{-1}\uproot{2}\scriptstyle \kappa}]{
        \frac{p_{\text{a}}}{p_{\text{b}}}} -p_{\text{a}}\right)
        \frac{p_0}{p_a} \frac{T_2}{T_0} \frac{1}{\rho_0}
        \frac{\symup{d}m}{\symup{dt}}
\end{equation}
Die Unsicherheit ist gegeben nach:
\begin{equation}
    \symup{\Delta}N_{\text{mech}} = \frac{1}{\kappa - 1} \left( p_{\text{b}}
        \sqrt[{\leftroot{-1}\uproot{2}\scriptstyle \kappa}]{
        \frac{p_{\text{a}}}{p_{\text{b}}}} -p_{\text{a}}\right)
        \frac{p_0}{p_a} \frac{T_2}{T_0} \frac{1}{\rho_0}
        \symup{\Delta}\frac{\symup{d}m}{\symup{dt}}
\end{equation}
%
\begin{table}[H]
    \centering
    \caption{Mechanische Kompressorleistung.}
    \label{tab:mech}
    \sisetup{table-format=2.2}
    \begin{tabular}{S[table-format=2.0] S[table-format=1.1] S[table-format=2.1] S[table-format=+1.3(4)] S[table-format=+1.3(4)]}
        \toprule
        {Zeit$/\si{\minute}$} & {$p_a/\si{\bar}$} & {$p_b/\si{\bar}$} & {$\frac{\symup{d}m}{\symup{dt}}/\SI{e-3}{\kg\per\second}$} & {$N_\text{mech}/\SI{e-3}{\watt}$} \\
        \midrule
        3   & 1.8   & 7.0   & -1.790\pm0.235    & -0.340\pm0.045 \\
        7   & 2.1   & 8.1   & -2.740\pm0.388    & -0.523\pm0.074 \\
        11  & 2.2   & 9.6   & -2.610\pm0.590    & -0.545\pm0.123 \\
        15  & 2.2   & 11.2  & -1.400\pm0.862    & -0.323\pm0.199 \\
        \bottomrule
    \end{tabular}
\end{table}
