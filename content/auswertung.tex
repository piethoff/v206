\section{Auswertung}
\label{sec:Auswertung}
\subsection{Temperaturverlauf}
Die gemessenen Werte sind in einem geeigneten Diagramm \ref{fig:tempfit} dargestellt.
Die Ausgleichsrechnung wurde in diesem Diagramm bereits aufgetragen.
\begin{table}[H]
    \centering
    \caption{Messwerte des Versuchs (ohne 1 Bar Umgebungsdruck).}
    \label{tab:t1}
    \sisetup{table-format=2.2}
    \begin{tabular}{S[table-format=2.0] S S S S S[table-format=3.0]}
        \toprule
        {Zeit$/\si{\minute}$} & {$T_1/\si{\celsius}$} & {$T_2/\si{\celsius}$} & {$p_a/\si{\bar}$} & {$p_b/\si{\bar}$} & {$A/\si{\watt}$}\\
        \midrule
        1   & 21.6  & 21.0  & 1.4   & 6     & 160   \\
        2   & 22.3  & 21.0  & 1.6   & 6.2   & 175   \\
        3   & 23.5  & 20.0  & 1.82  & 7.0   & 180   \\
        4   & 25.0  & 18.7  & 2.0   & 7.0   & 195   \\
        5   & 26.8  & 17.0  & 2.0   & 7.2   & 197   \\
        6   & 28.5  & 15.2  & 2.1   & 8.0   & 201   \\
        7   & 30.4  & 13.4  & 2.1   & 8.1   & 202   \\
        8   & 32.4  & 11.5  & 2.2   & 8.5   & 205   \\
        9   & 34.7  & 9.7   & 2.2   & 9     & 210   \\
        10  & 35.9  & 8.0   & 2.2   & 9.5   & 210   \\
        11  & 37.8  & 6.3   & 2.2   & 9.6   & 212   \\
        12  & 39.3  & 4.7   & 2.2   & 10.1  & 210   \\
        13  & 40.8  & 3.3   & 2.2   & 10.5  & 212   \\
        14  & 42.2  & 2.0   & 2.2   & 11.0  & 214   \\
        15  & 44.1  & 0.9   & 2.2   & 11.2  & 215   \\
        16  & 45.7  & 0.2   & 2.2   & 11.5  & 215   \\
        17  & 47.2  & -0.3  & 2.2   & 11.6  & 210   \\
        \bottomrule
    \end{tabular}
\end{table}
\subsection{Ausgleichsrechnung}
Um eine geeignete Näherung für die Temperaturverläufe zu erreichen wurde
\begin{equation}
  \label{eq:tempfit}
  T(t) = At^2 + Bt + C
\end{equation}
gewählt.
\begin{figure}[H]
  \centering
  \includegraphics[scale=0.5]{tempfit.pdf}
  \caption{Temperaturen der Wärmereservoirs aufgetragen gegen die Zeit}
  \label{fig:tempfit}
\end{figure}
Dabei ergeben sich für die Koeffizienten $A$, $B$ und $C$:

Die Differentialquotienten $\symup{d}T_1/\symup{dt}$ und $\symup{d}T_2/\symup{dt}$ können mithilfe der Näherungsfunktion 
bestimmt werden:
\begin{equation}
    \frac{\symup{d}T_1}{\symup{dt}} = 2At + B
\end{equation}
Es ergeben sich folgende Werte für 4 verschiedene, gewählte Temperaturen:
\begin{table}[H]
    \centering
    \caption{Differentialquotienten von $T_1$ und $T_2$.}
    \label{tab:t2}
    \sisetup{table-format=}
    \begin{tabular}{S S S S S}
        \toprule
        {$T_1/\si{\kelvin}$} & {$\frac{\symup{d}T_1}{\symup{dt}}$} & {$T_2/\si{\kelvin}$} & {$\frac{\symup{d}T_2}{\symup{dt}}$} & {$A/\si{\watt}$} \\
        \midrule
%
        \bottomrule
    \end{tabular}
\end{table}
Mithilfe der Gleichungen \eqref{} und \eqref{} kann nun die Gütezahl der realen Wärmepumpe berechnet werden:
\begin{equation}
    \nu_\text{real} = \frac{m_wc_w + m_sc_s}{A}\frac{\symup{d}T}{\symup{dt}}
\end{equation}
Hier ist $A$ die angezeigte Leistung des Kompressors, $m_sc_s$ die Wärmekapazität des Systems angegeben mit 
$\SI{660}{\joule\per\kelvin}$ und $m_wc_w$ die Wärmekapazität des Wassers in einem der Reservoire, die mit dem Literaturwert 
von $c_w = \SI{4182}{\joule\per\kilogram\per\kelvin}$ bestimmt wird.
Somit ergibt sich eine Güte von $\nu_\text{real} = $
Im Vergleich zur Güteziffer der idealen Wärmepumpe $\nu_\text{ideal} = $ nach \eqref{} hat die reale Waärmepumpe eine 
Abweichung von \%.
