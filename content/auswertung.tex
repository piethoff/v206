\section{Auswertung}
\label{sec:Auswertung}
\subsection{Temperaturverlauf}
Die gemessenen Werte sind in einer geeigneten Abbildung \ref{fig:tempfit} dargestellt.
Die Ausgleichsrechnung wird in diesem Diagramm bereits aufgetragen.
\begin{table}[H]
    \centering
    \caption{Messwerte des Versuchs (ohne 1 Bar Umgebungsdruck).}
    \label{tab:t1}
    \sisetup{table-format=2.2}
    \begin{tabular}{S[table-format=2.0] S S S S S[table-format=3.0]}
        \toprule
        {Zeit$/\si{\minute}$} & {$T_1/\si{\celsius}$} & {$T_2/\si{\celsius}$} & {$p_a/\si{\bar}$} & {$p_b/\si{\bar}$} & {$A/\si{\watt}$}\\
        \midrule
        1   & 21.6  & 21.0  & 1.4   & 6.0   & 160   \\
        2   & 22.3  & 21.0  & 1.6   & 6.2   & 175   \\
        3   & 23.5  & 20.0  & 1.8   & 7.0   & 180   \\
        4   & 25.0  & 18.7  & 2.0   & 7.0   & 195   \\
        5   & 26.8  & 17.0  & 2.0   & 7.2   & 197   \\
        6   & 28.5  & 15.2  & 2.1   & 8.0   & 201   \\
        7   & 30.4  & 13.4  & 2.1   & 8.1   & 202   \\
        8   & 32.4  & 11.5  & 2.2   & 8.5   & 205   \\
        9   & 34.7  & 9.7   & 2.2   & 9.0   & 210   \\
        10  & 35.9  & 8.0   & 2.2   & 9.5   & 210   \\
        11  & 37.8  & 6.3   & 2.2   & 9.6   & 212   \\
        12  & 39.3  & 4.7   & 2.2   & 10.1  & 210   \\
        13  & 40.8  & 3.3   & 2.2   & 10.5  & 212   \\
        14  & 42.2  & 2.0   & 2.2   & 11.0  & 214   \\
        15  & 44.1  & 0.9   & 2.2   & 11.2  & 215   \\
        16  & 45.7  & 0.2   & 2.2   & 11.5  & 215   \\
        17  & 47.2  & -0.3  & 2.2   & 11.6  & 210   \\
        18  & 48.6  & -0.7  & 2.2   & 12.0  & 210   \\
        \bottomrule
    \end{tabular}
\end{table}
\subsection{Ausgleichsrechnung}
Als geeignete Näherung für die Temperaturverläufe wird
\begin{equation}
    T \approx F(t) = At^3 + Bt^2 + Ct + D
\end{equation}
gewählt.
\begin{figure}[H]
  \centering
  \includegraphics[scale=0.5]{tempfit.pdf}
  \caption{Temperaturen der Wärmereservoirs aufgetragen gegen die Zeit.}
  \label{fig:tempfit}
\end{figure}
\noindent Dabei ergeben sich für die Koeffizienten $A$, $B$ und $C$:
\begin{align*}
    A_{T1} & = \SI[per-mode=reciprocal]{-1.3 \pm 0.28e-8}{\kelvin\per\second\cubed}    \\
    B_{T1} & = \SI[per-mode=reciprocal]{2.2 \pm 0.5e-5}{\kelvin\per\second\squared}           \\
    C_{T1} & = \SI[per-mode=reciprocal]{0.018 \pm 0.024}{\kelvin\per\second}                       \\
    D_{T1} & = \SI[per-mode=reciprocal]{293.18 \pm 0.33}{\kelvin}                       \\
\\
    A_{T2} & = \SI[per-mode=reciprocal]{3.39 \pm 0.23e-8}{\kelvin\per\second\cubed}    \\
    B_{T2} & = \SI[per-mode=reciprocal]{-5.2 \pm 0.4e-5}{\kelvin\per\second\squared}           \\
    C_{T2} & = \SI[per-mode=reciprocal]{-0.004 \pm 0.002}{\kelvin\per\second}                       \\
    D_{T2} & = \SI[per-mode=reciprocal]{295.05 \pm 0.27}{\kelvin}                       \\
\end{align*}
Die Differentialquotienten $\symup{d}T_1/\symup{dt}$ und $\symup{d}T_2/\symup{dt}$ können mithilfe der Näherungsfunktion
bestimmt werden:
\begin{equation}
    \frac{\symup{d}T}{\symup{dt}} \approx 3At^2 + 2Bt + C
\end{equation}
Die Unsichertheit ergibt sich mit der Gaußschen Fehlerfortpflanzung zu:
\begin{equation}
    \symup{\Delta}\frac{\symup{d}T}{\symup{dt}} = \sqrt{\left(3t^2\symup{\Delta}A\right)^2+\left(2t\symup{\Delta}B\right)^2+\left(\symup{\Delta}C\right)^2}
\end{equation}
Es ergeben sich folgende Werte für 4 verschiedene, gewählte Temperaturen:
\begin{table}[H]
    \centering
    \caption{Differentialquotienten von $T_1$ und $T_2$.}
    \label{tab:t2}
    \sisetup{table-format=3.4(4)e1}
    \begin{tabular}{S[table-format=2.0] S[table-format=3.2] S[table-format=1.3(4)] S[table-format=3.2] S[table-format=2.3(4)]}
        \toprule
        {Zeit$/\si{\minute}$} & {$T_1/\si{\kelvin}$} & {$\frac{\symup{d}T_1}{\symup{dt}}/\si{\kelvin\per\second}$} & {$T_2/\si{\kelvin}$} & {$\frac{\symup{d}T_2}{\symup{dt}}/\si{\kelvin\per\second}$} \\
        \midrule
        3   & 296.65    & 0.028 \pm 0.004 &  293.15 & -0.031 \pm 0.009\\
        7   & 303.55    & 0.028 \pm 0.010 &  286.55 & -0.031 \pm 0.020\\
        11  & 310.95    & 0.028 \pm 0.015 &  279.45 & -0.031 \pm 0.032\\
        15  & 317.25    & 0.028 \pm 0.021 &  274.05 & -0.031 \pm 0.044\\
        \bottomrule
    \end{tabular}
\end{table}
%
\subsection{Güteziffer}
Mithilfe der Gleichungen \eqref{eq:gueteziffer} und \eqref{eq:leistung} kann nun die Gütezahl der realen Wärmepumpe berechnet
werden:
\begin{equation}
    \nu_\text{real} = \frac{m_wc_w + m_sc_s}{N}\frac{\symup{d}T}{\symup{dt}}
\end{equation}
Die Unsicherheit ist bestimmt durch:
\begin{equation}
    \symup{\Delta}\nu_\text{real} = \frac{m_wc_w + m_sc_s}{N}\symup{\Delta}\frac{\symup{d}T}{\symup{dt}}
\end{equation}
Hier ist $N$ die angezeigte Leistung des Kompressors, $m_sc_s$ die Wärmekapazität des Systems angegeben mit
$\SI{660}{\joule\per\kelvin}$ und $m_wc_w$ die Wärmekapazität des Wassers in einem der Reservoire, die mit dem Literaturwert
von \mbox{$c_w = \SI{4182}{\joule\per\kilogram\per\kelvin}$ \cite{const}} bestimmt wird.
Die so erhaltenen Güteziffern befindent sich in Tabelle \ref{tab:guete}.
Dazu sind noch die idealen Güteziffern und Abweichung von diesen aufgetragen.
\begin{table}[H]
    \centering
    \caption{Ergebnisse für reale und ideale Güteziffern.}
    \label{tab:guete}
    \sisetup{table-format=2.2}
    \begin{tabular}{S[table-format=2.0] S[table-format=1.3(4)] S S}
        \toprule
        {Zeit$/\si{\minute}$} & {$\nu_\text{real}$} & {$\nu_\text{ideal}$} & {Abweichung} \\
        \midrule
        3  & 0.205 \pm 0.290 & 84.76 & 99.76\% \\
        7  & 0.183 \pm 0.662 & 17.86 & 98.97\% \\
        11 & 0.174 \pm 0.923 & 9.87  & 98.23\% \\
        15 & 0.172 \pm 1.274 & 7.34  & 97.66\% \\
        \bottomrule
    \end{tabular}
\end{table}
%
\subsection{Massendurchsatz}
Der Massendurchsatz lässt sich mit Gleichung \eqref{eq:massendurchsatz} berechnen, jedoch wird dazu die Verdampfungswärme
$L$ benötigt.
In der Anleitung des Versuchs \href{http://129.217.224.2/HOMEPAGE/PHYSIKER/BACHELOR/AP/SKRIPT/V203.pdf}{v203} findet sich
folgende Gleichung:
\begin{equation}
    \ln \left(p\right) = -\frac{L}{\symup{R}} \frac{1}{T} + \text{const.}
\end{equation}
In folgendem halblogarithmischen Diagramm, in welchem $\ln{p}$ gegen $\frac{1}{T}$ 
aufgetragen ist, ist eine Ausgleichsgerade eingefügt aus dessen Steigung m sich L berechnen 
lässt.
\begin{equation}
    L = -\frac{m}{R}
\end{equation}
Die Unsicherheit ist somit:
\begin{equation}
    \symup{\Delta}L = -\frac{\symup{\Delta}m}{R}
\end{equation}
\begin{figure}[H]
    \centering
    \includegraphics[width=\textwidth]{build/dampfdruck.pdf}
    \caption{Dampfdruckkurve von $\ce{Cl2F2C}$.}
\end{figure}
%
Aus dieser Gleichung und den Messreihen für $p_b$ und $T_1$ kann mit der allgemeinen 
Gaskonstante \mbox{$R=\SI{8.31}{\joule\per\mole\per\kelvin}$}
die Verdampfungswärme \mbox{$L=\SI{17301\pm548}{\joule\per\mole}=\SI{143101\pm4568}{\joule\kilogram}$}, mit der molaren
Masse von $\ce{Cl2F2C}$
\mbox{$\SI{120.91}{\gram\per\mole}$ \cite{molar}}, bestimmt werden.
%
Der Massendurchsatz kann nun mithilfe von Gleichung \eqref{dubbi9} und \eqref{dubbi11} 
zu
\begin{equation}
    \frac{\symup{d}m}{\symup{dt}} = \frac{1}{L}(m_2c_w + m_sc_s)\frac{\symup{d}T_2}{\symup{dt}}
\end{equation}
bestimmt werden mit der Unsicherheit
\begin{equation}
    \symup{\Delta}\frac{\symup{d}m}{\symup{dt}} = 
    \sqrt{
        \left(\frac{1}{L^2}(m_2c_w + m_sc_s)\frac{\symup{d}T_2}{\symup{dt}}\symup{\Delta}L\right)^2 + 
        \left(\frac{1}{L}(m_2c_w + m_sc_s)\symup{\Delta}\frac{\symup{d}T_2}{\symup{dt}}\right)^2
    }
\end{equation}
und sind im Folgenden aufgetragen:
\begin{table}[H]
    \centering
    \caption{Massendurchsatz.}
    \label{tab:masse}
    \sisetup{table-format=2.3(4)e2}
    \begin{tabular}{S[table-format=2.0] S S}
        \toprule
        {Zeit$/\si{\minute}$} & {$\frac{\symup{d}T_2}{\symup{dt}}/\si{\kelvin\per\second}$} & {$\frac{\symup{d}m}{\symup{dt}}/\si{\kg\per\second}$} \\
        \midrule
        3   & -0.031 \pm 0.009  & -2,861\pm0.836e-3\\
        5   & -0.031 \pm 0.020  & -2,861\pm1.848e-3\\
        7   & -0.031 \pm 0.032  & -2,861\pm2.955e-3\\
        11  & -0.031 \pm 0.044  & -2,861\pm4.062e-3\\
        \bottomrule
    \end{tabular}
\end{table}
%
\subsection{Mechanische Kompressorleistung}
Abschließend lässt sich die mechanische Kompressorleistung über Gleichung \eqref{eq:arbeit} bestimmen.
Die Unsicherheit ist gegeben nach:
\begin{equation}
    \symup{\Delta}N_{\text{mech}} = \frac{1}{\kappa - 1} \left( p_{\text{b}}
        \sqrt[{\leftroot{-1}\uproot{2}\scriptstyle \kappa}]{
        \frac{p_{\text{a}}}{p_{\text{b}}}} -p_{\text{a}}\right)
        \frac{1}{\rho}
        \symup{\Delta}\frac{\symup{d}m}{\symup{dt}}
\end{equation}
Die Dichte von $\ce{Cl2F2C}$ bei \mbox{$T=\SI{0}{\celsius}$} und \mbox{$p=\SI{1}{\bar}}$ liegt bei 
\mbox{$\rho_0=\SI{5.51}{\gram\per\liter}$ \cite{Cl2F2C}} und
$\kappa$ ist gegeben mit $\SI{1.14}{}$.
%
\begin{table}[H]
    \centering
    \caption{Mechanische Kompressorleistung.}
    \label{tab:masse}
    \sisetup{table-format=2.2}
    \begin{tabular}{S[table-format=2.0] S S S[table-format=2.3(4)e2] S[table-format=2.3(4)]}
        \toprule
        {Zeit$/\si{\minute}$} & {$p_a/\si{\bar}$} & {$p_b/\si{\bar}$} & {$\frac{\symup{d}m}{\symup{dt}}/\si{\kg\per\second}$} & {$N_\text{mech}/\si{\watt}$} \\
        \midrule
        3   & 1.82  & 7.0   & -2,861\pm0.836e-3 & -1.094\pm0.002 \\
        7   & 2.1   & 8.1   & -2,861\pm1.848e-3 & -1.265\pm0.002 \\
        11  & 2.2   & 9.6   & -2,861\pm2.955e-3 & -1.476\pm0.002 \\
        15  & 0.9   & 11.2  & -2,861\pm4.062e-3 & -1.233\pm0.002 \\
        \bottomrule
    \end{tabular}
\end{table}
